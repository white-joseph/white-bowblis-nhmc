%\documentclass{article}
\documentclass[preprint,authoryear]{elsarticle}

\usepackage{amsmath,amssymb}
\usepackage[margin=1.5in, top=1in, bottom=1in]{geometry}
\usepackage{booktabs}
\usepackage{tabularx}
\usepackage{threeparttable}
\usepackage{array}
\usepackage{caption}
\usepackage{subcaption}
\usepackage{makecell}
\usepackage{graphicx}
\usepackage[section]{placeins}
\usepackage{pdflscape}
\usepackage{newtxtext}
\usepackage{newtxmath}
\usepackage{soul,color}
\usepackage[hidelinks]{hyperref}

\captionsetup{labelfont=bf, font=small}
\newcolumntype{Y}{>{\centering\arraybackslash}X}
\newcommand{\sym}[1]{\rlap{$^{#1}$}}
\newcommand{\est}[3]{\makecell[c]{#1\sym{#3}\\ \footnotesize(#2)}}

\journal{Journal of Health Economics}

\begin{document}

\begin{frontmatter}
\title{My Title}
\author[miami]{Joe White\corref{cor1}}
\ead{whitejf@miamioh.edu}

% \author[miami]{John Bowblis\corref{cor2}}
% \ead{bowblijr@miamioh.edu}

\address[miami]{Department of Economics, Miami University, Oxford, OH, USA}

\cortext[cor1]{Corresponding author}
% \cortext[cor2]{Co-corresponding author}

\begin{abstract}

Ownership generally $\rightarrow$ Non-price/labor outcomes $\rightarrow$ Healthcare $\rightarrow$ Nursing homes $\rightarrow$ My paper

\end{abstract}

% \begin{keyword}
% health economics \sep event study \sep nursing homes
% \end{keyword}
\end{frontmatter}

\section{Introduction}

Ownership transitions, through acquisitions, sales, and mergers have become increasingly common across many sectors of the economy, especially in healthcare. There is extensive industrial organization research that focuses on ownership changes, motivated by concerns over market power, competition, and consumer welfare. Much of this literature focuses on mergers and acquisitions and how they affect prices, costs, and market structure, finding that they can substantially reshape competitive conditions across many different industries. This research has generated important insights into these areas and has played a central role in informing antitrust policy, specifically around mergers and acquisitions. In focusing primarily on prices and costs, there is a limited understanding about how ownership changes and transitions impact firms along other outcomes. In particular, less is known about how changes in ownership influence internal firm decisions and inputs to production that can have economic implications extending beyond prices and costs. As a result, while existing literature gives us a good framework for understanding the competitive effects ownership changes, there are important questions to be answered about non-price outcomes. \\

Much of the existing literature on ownership changes examines how they affect prices, costs, and market power, while little is known about how ownership changes influence inputs to labor supply that directly influence patient outcomes. Existing studies that do focus on non-price outcomes are limited and focus on outcomes such as innovation and sustainability \citep{haucap_stiebale_2023}, product characteristics \citep{fan_2013}, X,Y, and Z. There is also little evidence on how ownership changes impact internal decisions for supply of labor. Literature that does look at labor outcomes suggest a significant decline in employment and wages if an establishment undergoes and ownership change compared to those who do not \citep{lichtenberg_siegel_1990}. In their study, Lichtenberg and Siegel use data on employees who primarily work in business administration and research and development roles. The data for their study is decently outdated though, as their study period takes place from 1977 through 1982. There have been many major events and labor market transformations that would likely have changed labor market conditions between their study period and the one we use for our paper. Nevertheless, they find that employment declines by roughly 16 percent following an ownership change. One piece of the study that is similar to ours and is to our benefit is that it looks at ownership changes in general, and not just mergers and acquisitions. Where we aim to contribute to the literature is using updated data along with newer empirical techniques that this study does not utilize. Another paper that studies the affects of ownership changes on labor outcomes finds mixed results \citep{mcguckin_nguyen_2001}. They find that following an ownership change are not a way for firms to cut jobs and that typical ownership changes actually increase employment. The difference between this study and the earlier one by Lichtenberg and Siegel is that this looks at blue collar workers rather than white collar ones. These labor adjustments in response to ownership changes may be especially important in sectors where labor inputs are tightly linked to service quality, such as healthcare. \\

Within healthcare markets ownership changes can have particularly important economic and policy relevant consequences. Healtchare providers operate under extensive regulation, serve vulnerable populations, and rely on very skilled labor. All of these characteristics distinguish healthcare from many of the other industries studied in the merger and ownership change literature. Unlike settings where prices and ouputs are free to adjust, healthcare markets are often characterized by administered prices and regulatory constraints that can limit price competetition. As a result, changes in ownership may not only alter pricing or financial performance, but also internal organizational decisions that have a direct impact on the quality and accessibility of care. Literature on ownership changes in the healthcare industry are limited, and instead study markets such as craft beer \citep{fan_yang_2022} etc... Studies that do focus on healthcare markets study things such as pricing, utilization, spending, market power, and is usually focused on physician and hospital markets. \hl{examples}. Evidence on non-price outcomes is limited and offers mixed results, \hl{examples}. \\

Nursing homes provide a particularly important setting in which to study the staffing effects of ownership change. They serve a medically vulnerable population, operate under heavy regulatory oversight, and rely intensively on skilled and semi-skilled nursing labor. Changes in ownership may alter managerial objectives, financial constraints, and labor practices in ways that directly affect staffing decisions. These dynamics are especially relevant for policymakers and regulators concerned with access to care, workforce stability, and patient outcomes in long-term care markets. Nursing labor represents one of the largest operating expenses for nursing homes and is a primary determinant of care quality, safety, and resident outcomes. Staffing is a central input in the production of nursing home care and is extremely important to patient outcomes. There are many different patient outcomes that staffing is central to, the first being mortality. There are significant negative effects to inpatient mortality with shortages and decreases in university degree holding nurse staff \citep{kelly_et_al_2026}. Another patient outcome that nurses play a key role in is avoidable critical conditions. There are statistically significant differences in patients avoiding critical conditions when they had adequate nurse staff vs when they did not \citep{udina_et_al_2025}. A systematic review found that for registered nurses specifically,across the board studies show that higher staffing lead to reduced mortality \citep{dallora_et_al_2022}. As a result, staffing levels are a natural margin along which facilities may adjust in response to changes in financial incentives, managerial priorities, or organizational structure. As discussed previously, labor market outcomes as a result of ownership changes are very understudied. The gap in the literature that we aim to fill is understanding how nurse staffing responds after an ownership change, something that has not yet been looked at. Understanding how staffing responds to major organizational changes is therefore critical for evaluating the consequences of consolidation and ownership transitions in long-term care.\\

Using detailed administrative data from the Centers for Medicare and Medicaid Services (CMS), we study the causal effect of ownership changes on nursing hours per patient day. Exploiting staggered ownership transitions and a difference-in-differences event study design, we document significant declines in staffing following ownership changes. These effects vary by staff type, pre vs post-pandemic, and baseline chain affiliation, highlighting the role of organizational structure and labor market conditions in shaping staffing adjustments after ownership transitions.\\

\section{Data and Methods}
\subsection{Data Sources}
We utilize three main sources of data: Nursing Home Compare (NHC) Archive data, Healthcare Cost Report Information System (HCRIS), and Payroll-based Journal (PBJ) data. The NHC Archive contains monthly facility level information on ownership, chain affiliation, certification status, and quality measures. The second is HCRIS, also known as the Medicare Cost Reports, which contains annual cost reports submitted by every Medicare certified nursing home in the United States. The cost reports contain information about costs, revenue, financial data for each reporting facility, and dates of ownership changes. Lastly, we use the PBJ data to obtain staffing information. The PBJ data contains the census and the number of hours worked by various types of NH staff on a daily basis.

\subsection{Sample selection}

Our sample is constructed at the facility-year-month level. We create our sample using the following criteria. First, we limit our sample to the 48 contiguous U.S. states from the period January 2017 to June 2024. Secondly, the NHC Archive data is matched to the HCRIS data on a unique nursing home provider number. Because HCRIS only contains information on freestanding nursing homes, our sample excludes hospital-based nursing homes. Hospital based nursing homes make up about 12 percent of all nursing homes in the United States. 

%Lastly, if the facilities in our sample have more than one ownership change during our sample period then we only focus on the first change \textbf{ASK DR BOWBLIS ABOUT THIS}

\subsection{Identifying Ownership Changes}

To our knowledge there are no datasets that easily identify ownership changes for all nursing homes, therefore we needed to identify them using a two step procedure. 

The first step was to use the HCRIS data which contains an indicator for an ownership change, and corresponding date, in each cost report after an ownership change. We use this as our flag for ownership changes during the study period. However, these indicators may capture administrative or legal changes that do not reflect meaningful transfers of control. Thus, to verify these changes, our second step was to validate each potential ownership change in the HRCIS data with the NHC Archive data. The NHC Archive data reports ownership shares for each owner associated with a facility on a monthly basis. One issue with the NHC Archive data is that it reports the direct owners and indirect owners of each NH. For example, a limited liability corporation may directly own the NH, but that limited liability corporation is owned by different individuals. Therefore, we identified ownership changes to occur among direct and indirect owners using the following criteria: 

First, we define a change in ownership when a majority of ownership shares turn over between month $t-1$ and month $t$. Intuitively, this captures transactions in which control of the facility changes hands, including cases where a new operator acquires most or all shares. Secondly, when ownership percentages are partially or completely missing, we approximate shares by normalizing any reported percentages and, where necessary, assigning equal weights across remaining owners. To reduce false positives from purely legal restructuring, we do not flag a change in ownership if at least 80 percent of the ownership stake remains within the same surname family (e.g., a change from “Smith Holdings LLC” to “Smith Family Partners LP”). 

Using these verified ownership changes, the analytic sample included Nursing Homes that either (i) have no ownership change in either source or (ii) have exactly one ownership change in each source with the CHOW date within six months of one another. For these Nursing Homes, we define the change in ownership month as the first month in which the NHC Archive data indicate a change. All dates are converted to calendar months, so changes occurring at the beginning or end of a month are treated identically. 

\subsection{Nursing Staff Levels}

Our main dependent variables are measures of nursing staff levels. The nursing staff types we use in our study are registered nurses (RNs), licensed practical nurses (LPNs), and certified nursing assistants (CNAs). RNs are the most highly trained nursing staff and are responsible for assessment, care planning, medication administration, and supervision of other nursing personnel. LPNs provide basic medical care under the supervision of RNs and physicians, including monitoring vital signs and administering certain treatments. CNAs provide the majority of direct daily care to residents, assisting with activities of daily living. Because these staff types differ in training, scope of practice, and compensation, changes in staffing composition may reflect different organizational responses to ownership changes. We therefore examine each staff category separately, as well as total nursing hours per patient day. 

Nursing staff levels were calculated from PBJ data, which reports the number of hours worked and the census for each type of nursing staff on a daily basis. We used this information to calculate nursing staff levels in hours per patient day (HPPD) as follows: for each facility $i$, calendar month $t$, and staff type, we first aggregate daily hours to the monthly level and then compute
\begin{equation}
  \text{HPPD}_{i,t} = \frac{H_{i,t}}{R_{i,t}},
\end{equation}
where $H_{i,t}$ is the total hours worked by the staff type during month $t$ and $R_{i,t}$ is the total resident days in that facility-month. Total HPPD is defined as the sum of RN, LPN, and CNA HPPD.

We construct two additional measures of data quality. First, we define a monthly coverage ratio
\[
  C_{i,t} = \frac{\text{DaysReported}_{i,t}}{\text{DaysInMonth}_t},
\]
where $\text{DaysReported}_{i,t}$ is the number of distinct days in month $t$ with complete PBJ records for facility $i$. This measure allows us to identify incomplete data in our sample and describe our data completeness. We use this measure for sample construction and descriptive purposes, excluding facilities and observations with low reporting coverage.

Second, we track gaps in the reporting for each facility. For each observation we compute \textit{gap}, defined as the number of calendar months since the previous PBJ observation for that facility. This captures discontinuities in reporting that may arise from administrative faults or temporary periods of not reporting PBJ system, for example during the COVID-19 period. Our baseline specification places no restriction on reporting gaps, however; we use the gap measure in robustness analyses to assess whether our results are sensitive to excluding observations following reporting gaps.

Finally, we exclude a small number of observations with implausible staffing levels: months in which RN and LPN HPPD are both zero, total HPPD is below 1.5 or above 12, or CNA HPPD exceeds 5.25. These observations are outliers that likely reflect reporting errors rather than genuine variation in staffing. These exclusion criteria align with the criteria outlined in the CMS NHC Technical Users Guide.

\subsection{Other Controls}

To account for other factors that are correlated with nursing staff levels, we included a number of control variables that are consistent with other studies (XXX). These control variables were measured at the facility-level and came from NHC and HCRIS. The following variables are used as controls in all specifications: Ownership type (government, non-profit, and for-profit), chain affiliation, total number of beds, occupancy rate, payer mix (percent of revenue coming from Medicare, and percent of revenue coming from Medicaid), and acuity. Our measure of acuity is case-mix provided by CMS in the NHC archive data. During our study period CMS altered the methodology used to construct case-mix indices. As a result, raw case-mix values are not directly comparable over time. To address this issue, we measure acuity using relative rankings rather than levels. Specifically, we construct case-mix quartiles within each state and calendar month.

\subsection{Analytic Approach}

In our empirical setting, nursing homes change ownership at different points in time. Recent literature shows that conventional two-way fixed effects (TWFE) estimators can produce biased estimates in staggered adoption settings when treatment effects are heterogeneous over time (Goodman-Bacon, 2021). In particular, TWFE estimates can place negative weights on certain comparisons and implicitly rely on already treated units as controls. To circumvent this, we employ a staggered adoption event study regression design. This approach allows us to estimate treatment effects before and after ownership changes while assessing pre-trends. Specifically we estimate the following event study regression:

\begin{equation}
\label{eq:es_twfe}
Y_{it}
=
\sum_{\tau = -24}^{24}
    \beta_{\tau}\,
    \mathbf{1}\!\left\{ \text{event\_time}_{it} = \tau \right\}
    \cdot \text{Treat}_{i}
+
X_{it}^{\prime}\gamma
+
\alpha_i
+
\lambda_t
+
\varepsilon_{it}.
\end{equation}

where $Y_{it}$ denotes nursing staff levels in HPPD for facility $i$ in in month $t$. The variable $\text{event\_time}_{it}$ measures the number of months relative to the month of ownership change. Thus, the coefficients $\beta_{\tau}$ capture the evolution of the nursing staff levels before and after ownership transitions. The vector $X_{it}$ includes time-varying facility characteristics, while $\alpha_i$ and $\lambda_t$ denote facility and month fixed effects, respectively. Identification comes from within-facility changes in staffing relative to the timing of ownership transitions, using facilities that have not yet experienced a change as controls.

In addition to estimating using levels, we also estimate specifications where the dependent variable is expressed in logarithms, log($Y_{it}$). Log specifications allow coefficients to be interpreted as approximate percentage changes in staffing and reduce sensitivity to differences in baseline staffing levels across facilities. For outcomes with zero staffing in a given month, log specifications are omitted.

While our main specification relies on the event-study design, we also present TWFE estimates to summarize the average post ownership change effect. We estimate the following regression:

\begin{equation}
  Y_{it}
  =
  \beta_{\text{post}}\,\textit{Post}_{it}
  +
  X_{it}^{\prime}\boldsymbol{\gamma}
  +
  \alpha_i
  +
  \lambda_t
  +
  \varepsilon_{it},
  \label{eq:twfe-post}
\end{equation}

where $Y_{it}$ denotes nurse staffing outcomes for staff types RN, LPN, CNA, or total HPPD for facility $i$ in month $t$. The indicator $\textit{Post}_{it}$ equals one for all months following an ownership change and zero otherwise. Facility fixed effects $\alpha_i$ absorb time invariant differences across nursing homes, while month fixed effects $\lambda_t$ control for common shocks to staffing over time.

The coefficient $\beta_{\text{post}}$ captures the average change in staffing following an ownership transition, relative to the pre-change period. All specifications include the same time varying covariates as in the event-study models, and standard errors are clustered at the facility level.

\subsubsection{Subgroup Analysis}

We further examine heterogeneity in the effects of ownership change with two further analyses: the COVID-19 pandemic and baseline chain affiliation.

First, we estimate separate models for the pre-pandemic period (January 2017 through December 2019) and the pandemic period (April 2020 through June 2024). Labor market conditions for nursing staff changed substantially during the COVID-19 pandemic, including heightened turnover, changes in wages, and staffing shortages. Estimating models separately by period allows us to assess whether ownership changes had differential effects on staffing before and during this structural shift in the nursing labor market.

Second, we examine heterogeneity by baseline chain status. We classify facilities based on whether they were affiliated with a chain at the beginning of the study period (January 2017), prior to any ownership transitions in our sample. Chain affiliated facilities differ systematically from independent facilities in ways that are directly relevant for staffing decisions. Prior work shows that chains operate internal labor markets, share managerial expertise across facilities, and are better positioned to reallocate staff in response to shocks, potentially lessening the effects of ownership changes on staffing outcomes. In contrast, independent facilities may face tighter local labor constraints and fewer organizational buffers, making staffing levels more sensitive to changes in ownership and management.

By looking at baseline chain status, we compare facilities with similar organizational structures prior to treatment, rather than confounding the effects of ownership change with changes in chain affiliation occurring at the same time. Estimating models separately for chain and non-chain facilities therefore allows us to assess whether ownership transitions have different effects on staffing across organizational structures, and whether chain affiliation dampens or magnifies staffing adjustments following ownership changes.

In both subgroup analyses, we estimate the same specifications as in equations~\eqref{eq:es_twfe} and~\eqref{eq:twfe-post}, maintaining consistent controls, fixed effects, and clustering. These analyses shed light on mechanisms and contextual factors that shape the staffing consequences of ownership changes.

\section{Results}

\subsection{Summary Statistics}

Table~\ref{tab:sumstats} reports summary statistics for the main analysis sample. The final sample consists of approximately 7,806 unique skilled nursing facilities and 632,231 facility-month observations spanning January 2017 through June 2024. Of these facilities,1589 experience an ownership change during the sample period.

\begin{table}[!ht]
\centering
\begin{threeparttable}
\caption{Summary Statistics}
\label{tab:sumstats}
\small
\setlength{\tabcolsep}{8pt}

\begin{tabularx}{\textwidth}{@{} l r r @{} }
\toprule
\textbf{Variable} & \textbf{Mean} & \textbf{SD} \\
\midrule
\multicolumn{3}{@{}l}{\textbf{Panel A: Outcome variables}} \\[2pt]
RN HPPD & 0.427 & 0.309 \\
LPN HPPD & 0.796 & 0.325 \\
CNA HPPD & 2.100 & 0.548 \\
Total HPPD & 3.324 & 0.793 \\
\addlinespace[0.8em]
\multicolumn{3}{@{}l}{\textbf{Panel B: Control variables}} \\[2pt]
Government (dummy) & 0.101 & 0.301 \\
Non-profit (dummy) & 0.171 & 0.377 \\
Chain affiliation (dummy) & 0.546 & 0.498 \\
Beds & 108.4 & 57.9 \\
Occupancy rate (\%) & 76.4 & 15.5 \\
\% Medicare & 13.0 & 11.1 \\
\% Medicaid & 60.5 & 20.2 \\
Acuity quartile 2 (state-month) & 0.231 & 0.422 \\
Acuity quartile 3 (state-month) & 0.230 & 0.421 \\
Acuity quartile 4 (state-month) & 0.233 & 0.423 \\
\bottomrule
\end{tabularx}

\begin{tablenotes}[flushleft]
\footnotesize
\item \textit{Notes:} Rows $=$ 632,231; Facilities $=$ 7,806; Treated facilities $=$ 1,589; Period $=$ 2017/01–2024/06; Average months per facility $=$ 81.0.
\end{tablenotes}

\end{threeparttable}
\end{table}

\FloatBarrier

\subsection{Parallel Trends}

The identifying assumption underlying our event-study design is that, absent an ownership change, staffing outcomes in treated facilities would have followed the same trends as those in untreated facilities. We assess the plausibility of this assumption by examining estimated coefficients on the pre-treatment event-time indicators and by conducting joint tests of pre-trends. Specifically, we test whether the coefficients on all pre-treatment leads of the ownership change are jointly equal to zero using Wald tests. 

Table~\ref{tab:pretrend-pvals-by-tau} reports p-values for individual pre-treatment event-time coefficients across outcomes, separately for specifications that include all observations (Panel~A) and for specifications that exclude the anticipatory period immediately preceding ownership changes (Panel~B). Under the parallel trends assumption, we would expect pre-treatment coefficients to be statistically indistinguishable from zero, implying relatively large p-values and no systematic pattern of significance across pre-treatment periods.

\input{outputs/tables/pretrend_wald_tests_fragment.tex}
\FloatBarrier

In Panel~A, which includes the full sample and uses $\tau=-1$ as the reference period, we observe widespread statistical significance in the pre-treatment coefficients for several staffing outcomes, particularly for registered nurses and total nursing hours. This indicates the presence of differential staffing trends prior to the ownership transition. The persistence and clustering of significant p-values well before the ownership change suggest that staffing adjustments begin before the formal transition date, consistent with anticipatory behavior by facilities.

Panel~B presents results from specifications that exclude the three months immediately preceding ownership changes ($\tau=-3,-2,-1$) and center the event study at $\tau=-4$. In contrast to Panel~A, the p-values in Panel~B are uniformly large showing no evidence of statistically significant pre-treatment differences. This pattern is consistent with parallel trends once anticipatory periods are excluded and provides support for the identifying assumption underlying our main event study estimates.

These results indicate that including anticipatory periods leads to violations of the parallel trends assumption, while excluding these periods gives results consistent with parallel pre-trends. Accordingly, our preferred specifications estimate equation~\eqref{eq:es_twfe} using only observations outside the anticipatory window and interpret dynamic treatment effects relative to $\tau=-4$. This restriction is motivated by evidence of staffing adjustments prior to formal ownership transitions, which would otherwise violate the parallel trends assumption. Standard errors are clustered at the facility level to account for serial correlation within nursing homes

\subsection{Main Results}

Given our main analysis suggests that parallel trends assumption likely holds, Figure~\ref{fig:event_study_main_landscape} presents event study estimates of the effect of ownership changes on nursing staffing outcomes (i.e., Equation~\ref{eq:es_twfe}). Following an ownership transition, staffing declines persist for many periods, with the largest and most precise effects occurring for registered nurses (RN), certified nursing assistants (CNA), and total nursing hours per patient day.

RN staffing declines immediately following ownership changes and remains far below pre-transition levels throughout the post-treatment period. CNA staffing exhibits a similar but slightly smaller decline, while total nursing hours mirror the combined reductions across staff types. In contrast, LPN staffing shows limited and statistically insignificant changes over time. These dynamic patterns indicate that ownership transitions lead to sustained reductions in staffing rather than short run adjustments.


\newgeometry{left=1.2cm,right=1.2cm,top=1.5cm,bottom=1.5cm}

\begin{landscape}
\begin{figure}[p]
\centering

% Row 1
\begin{subfigure}[t]{0.49\linewidth}
  \centering
  \includegraphics[width=\linewidth]{outputs/plots/twfe_es_rn_without_anticipation_drop_3_1_.pdf}
  \caption{Registered Nurses (RN)}
\end{subfigure}\hfill
\begin{subfigure}[t]{0.49\linewidth}
  \centering
  \includegraphics[width=\linewidth]{outputs/plots/twfe_es_lpn_without_anticipation_drop_3_1_.pdf}
  \caption{Licensed Practical Nurses (LPN)}
\end{subfigure}

\vspace{1em}

% Row 2
\begin{subfigure}[t]{0.49\linewidth}
  \centering
  \includegraphics[width=\linewidth]{outputs/plots/twfe_es_cna_without_anticipation_drop_3_1_.pdf}
  \caption{Certified Nursing Assistants (CNA)}
\end{subfigure}\hfill
\begin{subfigure}[t]{0.49\linewidth}
  \centering
  \includegraphics[width=\linewidth]{outputs/plots/twfe_es_total_without_anticipation_drop_3_1_.pdf}
  \caption{Total Nursing Hours}
\end{subfigure}

\caption{
Dynamic Effects of Ownership Change on Nursing Staffing.
Event study estimates of the effect of a change in ownership on nursing staff levels in hours per patient day (HPPD).
Points show coefficient estimates relative to the month immediately preceding the ownership change;
vertical bars denote 95\% confidence intervals.
The three months prior to the ownership change ($\tau=-3,-2,-1$) are excluded.
All specifications include control variables and facility and month fixed effects.
}
\label{fig:event_study_main_landscape}

\end{figure}
\end{landscape}
\FloatBarrier
\restoregeometry

To summarize these dynamic effects with a single average treatment estimate, Table~\ref{tab:twfe-post-full} reports two-way fixed effects estimates of the ownership change effect (i.e., Equation~\ref{eq:twfe-post}). Panel A reports results in levels, while Panel B reports logged results. Consistent with the event study results, ownership changes lead to meaningful reductions in staffing levels. RN staffing declines by approximately 0.03 hours per patient day (roughly 9 percent), CNA staffing falls by about 0.05 hours (approximately 3 percent), and total staffing decreases by 0.08 hours per patient day (roughly 3 percent). LPN staffing shows no statistically significant change. Estimates are nearly identical whether the observations for the three month period before the ownership change occurred is included or excluded from the model.

\begingroup
\begin{table}[!ht]
\centering
\begin{threeparttable}
\caption{Two-Way Fixed Effects Estimates of \textit{post} on Staffing Outcomes (Baseline)}
\label{tab:twfe-post-full}
\small
\setlength{\tabcolsep}{6pt}

\begin{tabularx}{\textwidth}{@{} l YYYY @{} }
\toprule
 & \multicolumn{4}{c}{\textbf{Outcomes}} \\
\cmidrule(lr){2-5}
 & \textbf{RN} & \textbf{LPN} & \textbf{CNA} & \textbf{Total} \\
\midrule
\multicolumn{5}{@{}l}{\textbf{Panel A: Staffing Levels in HPPD}} \\[2pt]
With anticipation  &  \est{$-0.031$}{$0.005$}{***}  &  \est{$\phantom{-}0.001$}{$0.006$}{}  &  \est{$-0.052$}{$0.009$}{***}  &  \est{$-0.082$}{$0.012$}{***} \\
Without anticipation  &  \est{$-0.032$}{$0.005$}{***}  &  \est{$\phantom{-}0.000$}{$0.006$}{}  &  \est{$-0.054$}{$0.009$}{***}  &  \est{$-0.086$}{$0.013$}{***} \\

\addlinespace[3pt]
\multicolumn{5}{@{}l}{\textbf{Panel B: Log Staffing Levels in HPPD}} \\[2pt]
With anticipation  &  \est{$-0.086$}{$0.016$}{***}  &  \est{$\phantom{-}0.004$}{$0.009$}{}  &  \est{$-0.026$}{$0.005$}{***}  &  \est{$-0.025$}{$0.004$}{***} \\
Without anticipation  &  \est{$-0.092$}{$0.016$}{***}  &  \est{$\phantom{-}0.003$}{$0.009$}{}  &  \est{$-0.028$}{$0.005$}{***}  &  \est{$-0.026$}{$0.004$}{***} \\
\bottomrule
\end{tabularx}

\begin{tablenotes}[flushleft]
\footnotesize
\item \textit{Notes:} Each cell reports the coefficient on \textit{post} with two-way clustered standard errors (by facility and month) in parentheses. Panel~A reports levels (HPPD); Panel~B reports logs (HPPD). Samples: \textit{With anticipation} ($N_{\mathrm{levels}}=632,231;\ N_{\mathrm{logs}}=632,231$). \textit{Without anticipation} ($N_{\mathrm{levels}}=628,107;\ N_{\mathrm{logs}}=628,107$).
\item All specifications include facility and month fixed effects and covariates: \textit{government}, \textit{non-profit}, \textit{chain}, \textit{beds}, \textit{occupancy rate}, \textit{percent Medicare}, \textit{percent Medicaid}, and state case-mix quartile indicators.
\item Statistical significance: $^{***}p<0.01$, $^{**}p<0.05$, $^{*}p<0.10$.
\end{tablenotes}
\end{threeparttable}
\end{table}
\endgroup


\FloatBarrier

\subsection{Effects Before and During Covid-19 Pandemic}

Table~\ref{tab:twfe-prepost} examines whether these effects differ before and after the COVID-19 pandemic. The magnitude of the decline is roughly halved in the post pandemic period for RN, CNA, and total HPPD, suggesting that pandemic induced shifts in labor markets and staffing substantially reduced the impact of ownership transitions.

\begingroup
\begin{table}[!ht]
\centering
\begin{threeparttable}
\caption{TWFE Estimates of \textit{post}: Pre- vs Post-pandemic Periods (Without anticipation)}
\label{tab:twfe-prepost}
\small
\setlength{\tabcolsep}{6pt}

\begin{tabularx}{\textwidth}{@{} l YYYY @{} }
\toprule
 & \multicolumn{4}{c}{\textbf{Outcomes}} \\
\cmidrule(lr){2-5}
 & \textbf{RN} & \textbf{LPN} & \textbf{CNA} & \textbf{Total} \\
\midrule
\multicolumn{5}{@{}l}{\textbf{Panel A: Staffing Levels in HPPD}} \\[2pt]
Pre-Pandemic Period (2017/01 - 2019/12) & \est{$-0.029$}{$0.006$}{***}  &  \est{$-0.016$}{$0.008$}{*}  &  \est{$-0.063$}{$0.013$}{***}  &  \est{$-0.108$}{$0.019$}{***} \\
Pandemic Period (2020/04 - 2024/06) & \est{$-0.018$}{$0.007$}{**}  &  \est{$-0.000$}{$0.008$}{}  &  \est{$-0.041$}{$0.015$}{***}  &  \est{$-0.058$}{$0.020$}{***} \\

\addlinespace[3pt]
\multicolumn{5}{@{}l}{\textbf{Panel B: Log Staffing Levels in HPPD}} \\[2pt]
Pre-Pandemic Period (2017/01 - 2019/12) & \est{$-0.068$}{$0.024$}{***}  &  \est{$-0.009$}{$0.012$}{}  &  \est{$-0.032$}{$0.006$}{***}  &  \est{$-0.033$}{$0.006$}{***} \\
Pandemic Period (2020/04 - 2024/06) & \est{$-0.063$}{$0.021$}{***}  &  \est{$\phantom{-}0.002$}{$0.011$}{}  &  \est{$-0.020$}{$0.008$}{**}  &  \est{$-0.019$}{$0.006$}{***} \\
\bottomrule
\end{tabularx}

\begin{tablenotes}[flushleft]
\footnotesize
\item \textit{Notes:} Each cell reports the coefficient on \textit{post} with two-way clustered standard errors (by facility and month) in parentheses. Panel~A reports levels (HPPD); Panel~B reports logs (HPPD). Sample sizes: Row~1 ($N_{\mathrm{levels}}=255,005;\ N_{\mathrm{logs}}=255,005$), Row~2 ($N_{\mathrm{levels}}=359,413;\ N_{\mathrm{logs}}=359,413$).
\item All specifications include facility and month fixed effects and covariates: \textit{government}, \textit{non-profit}, \textit{chain}, \textit{beds}, \textit{occupancy rate}, \textit{percent Medicare}, \textit{percent Medicaid}, and state case-mix quartile indicators.
\item Statistical significance: $^{***}p<0.01$, $^{**}p<0.05$, $^{*}p<0.10$.
\item Pre-pandemic 2017/01--2019/12; Pandemic 2020/04--2024/06.
\end{tablenotes}
\end{threeparttable}
\end{table}
\endgroup


\FloatBarrier

\subsection{Effects by Chain Status}

Table~\ref{tab:twfe-chain-nonchain} compares effects for facilities that began the study period as part of a chain and those that did not. Staffing reductions are noticeably larger among non chain facilities. CNA HPPD falls by 0.034 hours in chain facilities and by 0.079 hours in non chains, and total HPPD declines by 0.062 versus 0.113 hours, respectively. On a log scale, these differences translate into substantially larger percentage declines for non-chain facilities. LPN staffing again shows minimal movement.

\begingroup
\begin{table}[!ht]
\centering
\begin{threeparttable}
\caption{TWFE Estimates of \textit{post}: Chain vs Non-chain Facilities (Jan 2017 Baseline, Without anticipation)}
\label{tab:twfe-chain-nonchain}
\small
\setlength{\tabcolsep}{6pt}

\begin{tabularx}{\textwidth}{@{} l YYYY @{} }
\toprule
 & \multicolumn{4}{c}{\textbf{Outcomes}} \\
\cmidrule(lr){2-5}
 & \textbf{RN} & \textbf{LPN} & \textbf{CNA} & \textbf{Total} \\
\midrule
\multicolumn{5}{@{}l}{\textbf{Panel A: Staffing Levels in HPPD}} \\[2pt]
Chain January 2017 & \est{$-0.031$}{$0.007$}{***}  &  \est{$\phantom{-}0.003$}{$0.008$}{}  &  \est{$-0.034$}{$0.011$}{***}  &  \est{$-0.062$}{$0.015$}{***} \\
Non-chain January 2017 & \est{$-0.037$}{$0.009$}{***}  &  \est{$\phantom{-}0.003$}{$0.011$}{}  &  \est{$-0.079$}{$0.019$}{***}  &  \est{$-0.113$}{$0.025$}{***} \\

\addlinespace[3pt]
\multicolumn{5}{@{}l}{\textbf{Panel B: Log Staffing Levels in HPPD}} \\[2pt]
Chain January 2017 & \est{$-0.099$}{$0.021$}{***}  &  \est{$\phantom{-}0.008$}{$0.013$}{}  &  \est{$-0.019$}{$0.006$}{***}  &  \est{$-0.020$}{$0.005$}{***} \\
Non-chain January 2017 & \est{$-0.085$}{$0.030$}{***}  &  \est{$\phantom{-}0.013$}{$0.017$}{}  &  \est{$-0.037$}{$0.010$}{***}  &  \est{$-0.032$}{$0.008$}{***} \\
\bottomrule
\end{tabularx}

\begin{tablenotes}[flushleft]
\footnotesize
\item \textit{Notes:} Each cell reports the coefficient on \textit{post} with two-way clustered standard errors (by facility and month) in parentheses. Panel~A reports levels (HPPD); Panel~B reports logs (HPPD). Sample sizes: Row~1 ($N_{\mathrm{levels}}=322,100;\ N_{\mathrm{logs}}=322,100$), Row~2 ($N_{\mathrm{levels}}=243,036;\ N_{\mathrm{logs}}=243,036$).
\item All specifications include facility and month fixed effects and covariates: \textit{government}, \textit{non-profit}, \textit{chain}, \textit{beds}, \textit{occupancy rate}, \textit{percent Medicare}, \textit{percent Medicaid}, and state case-mix quartile indicators.
\item Statistical significance: $^{***}p<0.01$, $^{**}p<0.05$, $^{*}p<0.10$.
\item Baseline chain classification determined by facility status in January 2017.
\end{tablenotes}
\end{threeparttable}
\end{table}
\endgroup


\FloatBarrier

Overall, ownership changes lead to sizable and persistent reductions in nursing staffing, with effects concentrated among RN, CNA, and total hours per patient day. These reductions are larger prior to the COVID-19 pandemic and among facilities that were not part of a chain at baseline. Together, the event study and TWFE results indicate that ownership transitions systematically worsen staffing conditions in nursing homes rather than inducing temporary adjustments.

\subsection{Robustness}

This section presents a series of robustness checks designed to assess the validity of our identifying assumptions and the sensitivity of our results to alternative model choices. Across all exercises, the estimated effects of ownership change on staffing remain stable in sign, magnitude, and statistical significance. Most of these will have tables/plots that will go in the appendix.

\subsubsection{Assumptions}

In this section I want to present results for event study and TWFE models that rely on different assumptions than the one that I used in my main specification.


\subsubsection{Event Window and Anticipation}

Staffing decisions may adjust in anticipation of formal ownership transitions, potentially violating the parallel trends assumption if these periods are included. To address this concern, we estimate our event study and TWFE models using alternative event windows that exclude months immediately preceding the ownership change and show that the estimated post-treatment effects remain largely unchanged. Row 6 of Table~\ref{tab:twfe-robustness-summary} reports estimates of equation~\ref{eq:twfe-post} with a wider anticipatory period excluded from the sample. The results from row 6 show no differences in values or statistical significance. Similarly, row 7 reports estimates under a smaller anticipatory window that is excluded from the data, and we see no differences in the results.

\subsubsection{Gap Size}

Because staffing data are occasionally missing or reported with gaps, we examine whether our results are sensitive to the length of time between consecutive observations. Restricting the sample to facilities with smaller reporting gap shows that estimates are similar to those in the full sample, indicating that data gaps do not drive our findings. Row 2 of Table~\ref{tab:twfe-robustness-summary} reports our results for equation~\ref{eq:twfe-post} with the restriction that facilities with a reporting gap of larger than 6 are excluded from our sample. Compared to the baseline (row 1) this result is identical across staff types and statistical significance. The exact same result is true for row 3 which reports estimates for equation~\ref{eq:twfe-post} where facilities with a reporting gap of greater than 3 are removed. Row 4 and 5 report estimates where gap greater than 1 or equal to 0 respectively, are removed. Similarly, we see no differences in these results with respect to the baseline. 

\subsubsection{For Profit vs Not for Profit}

Ownership changes may have different implications for staffing depending on a facility's ownership type. We therefore estimate separate effects for for-profit and not-for-profit nursing homes.

\section{Discussion}

Our results give new evidence that ownership changes in nursing homes lead to statistically significant declines in staffing levels among RNs, CNAs, and total nurse staffing, which suggests that as new ownership takes over labor inputs are a key area of adjustment. In a sector that has many fixed prices and costs as well as extensive regular, the easiest margin for ownership to cut operating costs is staff. Figure~\ref{fig:event_study_main_landscape} shows that RN, CNA, and total nurse staffing sharply falls around 3 months before the transfer of ownership is filed and stays consistent at the decreased level of staffing.

\section{Appendix}

\begingroup
\begin{table}[!ht]
\centering
\begin{threeparttable}
\caption{TWFE estimates of \textit{post} (without anticipation).}
\label{tab:twfe-robustness-summary}
\small
\setlength{\tabcolsep}{6pt}

\begin{tabularx}{\textwidth}{@{} l c YYYY @{} }
\toprule
 &  & \multicolumn{4}{c}{\textbf{Outcomes}} \\
\cmidrule(lr){3-6}
 & \textbf{N} & \textbf{RN} & \textbf{LPN} & \textbf{CNA} & \textbf{Total} \\
\midrule
\multicolumn{6}{@{}l}{\textbf{Panel A: Staffing Levels in HPPD}} \\[2pt]
(1) Baseline  &  628,107  &  \est{$-0.032$}{$0.005$}{***}  &  \est{$\phantom{-}0.000$}{$0.006$}{}  &  \est{$-0.054$}{$0.009$}{***}  &  \est{$-0.086$}{$0.013$}{***} \\
(2) Sample excludes \textit{gap} $> 6$  &  627,923  &  \est{$-0.032$}{$0.005$}{***}  &  \est{$-0.000$}{$0.006$}{}  &  \est{$-0.054$}{$0.009$}{***}  &  \est{$-0.086$}{$0.013$}{***} \\
(3) Sample excludes \textit{gap} $> 3$  &  627,626  &  \est{$-0.032$}{$0.005$}{***}  &  \est{$-0.000$}{$0.006$}{}  &  \est{$-0.054$}{$0.009$}{***}  &  \est{$-0.086$}{$0.013$}{***} \\
(4) Sample excludes \textit{gap} $> 1$  &  623,585  &  \est{$-0.032$}{$0.005$}{***}  &  \est{$\phantom{-}0.000$}{$0.006$}{}  &  \est{$-0.055$}{$0.009$}{***}  &  \est{$-0.086$}{$0.013$}{***} \\
(5) Sample excludes \textit{gap} $> 0$  &  623,585  &  \est{$-0.032$}{$0.005$}{***}  &  \est{$\phantom{-}0.000$}{$0.006$}{}  &  \est{$-0.055$}{$0.009$}{***}  &  \est{$-0.086$}{$0.013$}{***} \\
(6) Drop $t \in \{-4,-3,-2,-1\}$  &  626,729  &  \est{$-0.032$}{$0.005$}{***}  &  \est{$\phantom{-}0.000$}{$0.006$}{}  &  \est{$-0.055$}{$0.009$}{***}  &  \est{$-0.086$}{$0.013$}{***} \\
(7) Drop $t \in \{-2,-1\}$ only  &  628,107  &  \est{$-0.032$}{$0.005$}{***}  &  \est{$\phantom{-}0.000$}{$0.006$}{}  &  \est{$-0.054$}{$0.009$}{***}  &  \est{$-0.086$}{$0.013$}{***} \\
(8) For-profit only  &  456,452  &  \est{$-0.028$}{$0.006$}{***}  &  \est{$-0.002$}{$0.006$}{}  &  \est{$-0.059$}{$0.010$}{***}  &  \est{$-0.089$}{$0.014$}{***} \\
(9) Non-profit only  &  107,984  &  \est{$-0.061$}{$0.029$}{**}  &  \est{$\phantom{-}0.017$}{$0.038$}{}  &  \est{$-0.029$}{$0.054$}{}  &  \est{$-0.074$}{$0.068$}{} \\

\addlinespace[3pt]
\multicolumn{6}{@{}l}{\textbf{Panel B: Log Staffing Levels in HPPD}} \\[2pt]
(1) Baseline  &  628,107  &  \est{$-0.092$}{$0.016$}{***}  &  \est{$\phantom{-}0.003$}{$0.009$}{}  &  \est{$-0.028$}{$0.005$}{***}  &  \est{$-0.026$}{$0.004$}{***} \\
(2) Sample excludes \textit{gap} $> 6$  &  627,923  &  \est{$-0.092$}{$0.016$}{***}  &  \est{$\phantom{-}0.003$}{$0.009$}{}  &  \est{$-0.028$}{$0.005$}{***}  &  \est{$-0.026$}{$0.004$}{***} \\
(3) Sample excludes \textit{gap} $> 3$  &  627,626  &  \est{$-0.092$}{$0.016$}{***}  &  \est{$\phantom{-}0.003$}{$0.009$}{}  &  \est{$-0.028$}{$0.005$}{***}  &  \est{$-0.026$}{$0.004$}{***} \\
(4) Sample excludes \textit{gap} $> 1$  &  623,585  &  \est{$-0.092$}{$0.016$}{***}  &  \est{$\phantom{-}0.004$}{$0.009$}{}  &  \est{$-0.028$}{$0.005$}{***}  &  \est{$-0.026$}{$0.004$}{***} \\
(5) Sample excludes \textit{gap} $> 0$  &  623,585  &  \est{$-0.092$}{$0.016$}{***}  &  \est{$\phantom{-}0.004$}{$0.009$}{}  &  \est{$-0.028$}{$0.005$}{***}  &  \est{$-0.026$}{$0.004$}{***} \\
(6) Drop $t \in \{-4,-3,-2,-1\}$  &  626,729  &  \est{$-0.092$}{$0.016$}{***}  &  \est{$\phantom{-}0.003$}{$0.009$}{}  &  \est{$-0.028$}{$0.005$}{***}  &  \est{$-0.026$}{$0.004$}{***} \\
(7) Drop $t \in \{-2,-1\}$ only  &  628,107  &  \est{$-0.092$}{$0.016$}{***}  &  \est{$\phantom{-}0.003$}{$0.009$}{}  &  \est{$-0.028$}{$0.005$}{***}  &  \est{$-0.026$}{$0.004$}{***} \\
(8) For-profit only  &  456,452  &  \est{$-0.091$}{$0.018$}{***}  &  \est{$-0.001$}{$0.009$}{}  &  \est{$-0.029$}{$0.005$}{***}  &  \est{$-0.028$}{$0.004$}{***} \\
(9) Non-profit only  &  107,984  &  \est{$-0.099$}{$0.079$}{}  &  \est{$-0.004$}{$0.082$}{}  &  \est{$-0.015$}{$0.024$}{}  &  \est{$-0.017$}{$0.020$}{} \\
\bottomrule
\end{tabularx}

\begin{tablenotes}[flushleft]
\footnotesize
\item \textit{Notes:} Each cell reports the \textit{post} coefficient with two-way clustered standard errors at the facility and month levels. Panel~A uses staffing levels (HPPD); Panel~B uses logs of HPPD.
\item All specifications include facility and month fixed effects and controls for ownership (government, non-profit, chain), beds, occupancy rate, payer mix (%Medicare, %Medicaid), and state case-mix quartiles. Each row corresponds to a different sample restriction or anticipation-window choice, as described in the row label. The $N$ column reports the number of facility-month observations used in that specification.
Significance: $^{***}p<0.01$, $^{**}p<0.05$, $^{*}p<0.10$.
\end{tablenotes}

\end{threeparttable}
\end{table}
\endgroup


\FloatBarrier

\begingroup
\begin{table}[!ht]
\centering
\begin{threeparttable}
\caption{Pre-treatment Event-Time Coefficient p-values by $\tau$ (Levels Only)}
\label{tab:pretrend-pvals-by-tau}
\small
\setlength{\tabcolsep}{6pt}

\begin{tabularx}{\textwidth}{@{} l YYYY @{} }
\toprule
 & \multicolumn{4}{c}{\textbf{Outcomes (HPPD)}} \\
\cmidrule(lr){2-5}
 & \textbf{RN} & \textbf{LPN} & \textbf{CNA} & \textbf{Total} \\
\midrule
\multicolumn{5}{@{}l}{\textbf{Panel A: With Anticipation}} \\[2pt]
$\tau=-24$ & 0.0009 & 0.3074 & 0.0226 & 0.0408 \\
$\tau=-23$ & 0.0299 & 0.2654 & 0.0270 & 0.0133 \\
$\tau=-22$ & 0.0575 & 0.2864 & 0.0504 & 0.0298 \\
$\tau=-21$ & 0.0418 & 0.3507 & 0.0475 & 0.0269 \\
$\tau=-20$ & 0.0063 & 0.2533 & 0.0112 & 0.0046 \\
$\tau=-19$ & 0.0522 & 0.1386 & 0.0058 & 0.0027 \\
$\tau=-18$ & 0.1171 & 0.0333 & 0.0110 & 0.0031 \\
$\tau=-17$ & 0.1677 & 0.0303 & 0.0106 & 0.0028 \\
$\tau=-16$ & 0.0625 & 0.0599 & 0.0898 & 0.0117 \\
$\tau=-15$ & 0.0438 & 0.0761 & 0.0597 & 0.0116 \\
$\tau=-14$ & 0.0333 & 0.0208 & 0.0199 & 0.0015 \\
$\tau=-13$ & 0.0020 & 0.0155 & 0.0031 & 0.0002 \\
$\tau=-12$ & 0.0002 & 0.0107 & 0.0018 & 0.0001 \\
$\tau=-11$ & 0.0001 & 0.0771 & 0.0154 & 0.0017 \\
$\tau=-10$ & 0.0006 & 0.0689 & 0.0105 & 0.0011 \\
$\tau=-9$ & 0.0007 & 0.3447 & 0.0309 & 0.0111 \\
$\tau=-8$ & 0.0015 & 0.3051 & 0.0429 & 0.0162 \\
$\tau=-7$ & 0.0063 & 0.4569 & 0.0643 & 0.0322 \\
$\tau=-6$ & 0.0076 & 0.3574 & 0.0241 & 0.0133 \\
$\tau=-5$ & 0.0007 & 0.7619 & 0.0530 & 0.0281 \\
$\tau=-4$ & 0.0002 & 0.6479 & 0.0578 & 0.0209 \\
$\tau=-3$ & 0.0056 & 0.8746 & 0.0783 & 0.0740 \\
$\tau=-2$ & 0.0969 & 0.9108 & 0.4561 & 0.4669 \\
$\tau=-1$ (Ref.) & \textit{Ref.} & \textit{Ref.} & \textit{Ref.} & \textit{Ref.} \\
\addlinespace[6pt]
\multicolumn{5}{@{}l}{\textbf{Panel B: Without Anticipation ($t=-3,-2,-1$ dropped)}} \\[2pt]
$\tau=-24$ & 0.5788 & 0.1295 & 0.4807 & 0.9552 \\
$\tau=-23$ & 0.5263 & 0.3203 & 0.4694 & 0.5222 \\
$\tau=-22$ & 0.3655 & 0.4165 & 0.6315 & 0.7009 \\
$\tau=-21$ & 0.3649 & 0.5216 & 0.5136 & 0.6343 \\
$\tau=-20$ & 0.8508 & 0.4178 & 0.1977 & 0.2468 \\
$\tau=-19$ & 0.2875 & 0.2010 & 0.2497 & 0.3249 \\
$\tau=-18$ & 0.1749 & 0.0649 & 0.3432 & 0.3081 \\
$\tau=-17$ & 0.0665 & 0.0382 & 0.3063 & 0.2788 \\
$\tau=-16$ & 0.1941 & 0.1063 & 0.9627 & 0.6704 \\
$\tau=-15$ & 0.1700 & 0.1182 & 0.7171 & 0.5494 \\
$\tau=-14$ & 0.2872 & 0.0382 & 0.6055 & 0.3096 \\
$\tau=-13$ & 0.8594 & 0.0277 & 0.1558 & 0.0719 \\
$\tau=-12$ & 0.5721 & 0.0192 & 0.1990 & 0.0483 \\
$\tau=-11$ & 0.5003 & 0.1086 & 0.3802 & 0.1784 \\
$\tau=-10$ & 0.7170 & 0.1133 & 0.2711 & 0.1365 \\
$\tau=-9$ & 0.9555 & 0.4310 & 0.5703 & 0.4842 \\
$\tau=-8$ & 0.7648 & 0.4378 & 0.5979 & 0.5589 \\
$\tau=-7$ & 0.3153 & 0.6498 & 0.7126 & 0.7953 \\
$\tau=-6$ & 0.2416 & 0.5228 & 0.5363 & 0.6464 \\
$\tau=-5$ & 0.9556 & 0.9283 & 0.9247 & 0.9748 \\
$\tau=-4$ (Ref.) & \textit{Ref.} & \textit{Ref.} & \textit{Ref.} & \textit{Ref.} \\
\bottomrule
\end{tabularx}

\begin{tablenotes}[flushleft]
\footnotesize
\item \textit{Notes:} Each cell reports the p-value for the event-time coefficient at $\tau$ in the TWFE event-study specification with two-way clustered standard errors (by facility and month).
\item Reference periods: Panel~A uses $\tau=-1$ as the reference; Panel~B uses $\tau=-4$ as the reference.
\item Sample sizes: Panel~A ($N=632,231$); Panel~B ($N=628,107$).
\item All specifications include facility and month fixed effects and covariates: \textit{government}, \textit{non-profit}, \textit{chain}, \textit{beds}, \textit{occupancy rate}, \textit{percent Medicare}, \textit{percent Medicaid}, and state case-mix quartile indicators.
\end{tablenotes}
\end{threeparttable}
\end{table}
\endgroup


\FloatBarrier

\nocite{*}
\bibliographystyle{elsarticle-harv}
\bibliography{refs}

\end{document}